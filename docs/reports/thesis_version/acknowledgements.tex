\chapter*{Acknowledgement}
\begin{center}
\textit{To my teachers—who sparked insight within me, making visible to the mind what was invisible to the eyes—I offer my deepest gratitude. }\\
\end{center}
I am especially thankful to my father, Dinesh Mamgain, who first introduced me to the power of mathematics as my personal tutor and has nurtured me with unwavering support to this day. My sincere thanks to Mayank Malhotra, a teacher, who revealed mathematics as a language of thought and inspired my pursuit of higher education. I am grateful to Dr. Taufiq Ahmed, whose teaching in statistical mechanics and electronics helped me transition from abstract thinking to the realm of physics.

I would like to thank Prof. Dr. Philipp Richter, who supported me generously, even beyond the academic setting in my master studies. I am also indebted to Prof. Martin Wendt, who guided my intuition on astronomical distance determination and mentored me in the development of scientific writing skills. Special appreciation goes to Prof. Dr. Achim Feldmeier, whose lectures in natural philosophy aligned my thinking with the clarity of scientific reasoning and deepened my understanding of mathematical modelling. I am also grateful to Dr. Jose Bogio, who gave me the opportunity to work as a lab assistant at AIP and introduced me to the practical world of lasers and on-chip spectrographs, building solid intuition on the physics of light.  I am deeply indebted to Hendrik at KPMG Berlin, under whose guidance I gained practical experience and insight into data pipeline concepts—skills that have proven invaluable to this research.

This thesis would not have been possible without the invaluable mentorship of Dr. Jesper Storm, my master thesis supervisor—the most humble person I have ever met—under whose guidance I elevated my understanding from theoretical concepts to astronomical applications. I am equally grateful to Prof. Maria-Rosa L. Cioni, my secondary thesis supervisor, whose openness and organisation support helped me navigate the learning process with confidence. My sincere gratitude to Dr. Barry Madore for the valuable communication on the research topic, as his scientific work served as the basis of this research thesis.  

I am also deeply grateful to my friends, who each contributed to this journey in their own meaningful way. Their companionship, encouragement, and insight have been a constant source of strength. I thank Pankaj Negi for fostering an intellectually rich environment during my formative years. I am grateful to Marco Floris and Mattia Toffano for accompanying me as close friends during the early days of my studies abroad. My heartfelt thanks to Selina Syed, Jakob Drews, Merlin Wagner, Jan Vincent and Maximilian Ueberschar for their unwavering support, which has felt as comforting and constant as that of family. I also thank Kuan Yu, Luzie Frietag, Aaraw Rauniyar, Chryssi Tzagkarakis, Partha Pritam Das, Ravi Shankar Chaurasia, Theodor Mc Carthy and Dipesh Sosa for cultivating an academic space where open intellectual discourse flourished, enriching our understanding of interdisciplinary concepts. And finally, to Joy Krecke, for sharing wisdom and offering a broader perspective on perception and communication. Together, over time, their presence in my life has deeply influenced my interdisciplinary learning and continuously inspired me to seek a deeper understanding of this existence.

Above all, I owe my deepest love and gratitude to my mother, Geeta Mamgain, whose unwavering devotion, quiet strength, and constant belief in me have been the foundation of everything I have achieved. Her nurturing presence and endless support played a profound role in shaping my self-confidence and determination. This work stands as much on her sacrifices as it does on my efforts. \\

\begin{flushright}
Shubham Mamgain \\
Potsdam, 2025 
\end{flushright}
