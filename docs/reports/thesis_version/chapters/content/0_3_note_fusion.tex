\begin{notebox}[sharp corners, width=\textwidth]{Nuclear Fusion Reactions in Main Sequence Star}

{\small Accumulation of Deuterium opens possibility for fusing with one more proton to form Helion (2 protons and 1 neutron). Rising temperature fuses two Helion atoms to form Helium (2 protons and 2 neutrons) and emits two Hydrogen atoms. This series of reactions from Hydrogen to Helium is called p-p chain (pp I branch) reaction. Other rarer reactions can also occur, for instance, pp II branch when Helion fuses with Helium to form Beryllium (4 protons and 3 neutrons). Then either Beryllium decays to Lithium (3 protons and 4 neutron), Lithium fuses with proton and breakdown into two Helium atoms. Otherwise, in pp III branch, Beryllium fuses with proton to form Boron (4 proton and 4 neuton) which is unstable and breaks down into two Helium. The most rare case is pp IV branch, when Helion captures a proton and form Helium. 
%Source: F. Reines, Ann. Rev. Nucl. Sci. 10 (1960) 1-26}
}

Proton-Proton Reactions: Efficient for 1 $M_0$ star's core. \\

\hspace{1cm} Branch I: 83.30\% Helium production, at 10-18 MK


\begin{center}
	\ce{^1H ->[p][e+,$\nu_e$]^2D ->[^1H][ $\gamma$ ]  ^3He ->[^3He][2 ^1^H] ^4He } 	
\end{center}
%\vspace{-3mm}

\hspace{5.6cm} 0.42 \hspace{0.7cm} 5.49 \hspace{0.7cm} 12.85  \hfill (Energy release in MeV)

\hspace{1cm} Branch II: 16.70 \% Helium production, at 18-25 MK

\begin{center}
	\ce{^4He ->[He][$\gamma$]^7Be ->[e-][ $\nu_e$]  ^7Li ->[H] 2 ^4He }	
\end{center}
\vspace{-2mm}
	\hspace{5.6cm} 1.59 \hspace{0.6cm} 0.861 \hspace{0.3cm} 17.35 \hspace{0.9cm} \hfill (Energy release in MeV)
	\vspace{1mm}

\hspace{1cm} Branch III: 0.12\% Helium production, above 25 MK

\begin{center}
	\ce{^4He ->[He][$\gamma$]^7Be ->[H][ $\gamma$]  ^8B ->[][e+,$\nu$_e] ^8Be -> 2^4He }	
\end{center}
\vspace{-2mm}
	\hspace{4.9cm} 1.59 \hspace{0.6cm} 5.59 \hspace{1.7cm}     23.29   \hfill (Energy release in MeV)


\hspace{1cm} Branch IV: 0.00002\% Helium generation in Sun 

\begin{center}
	\ce{^3He ->[H][e+,$\nu$_e] ^4He}	
\end{center}
	\hspace{7.1cm} 19.795 \hspace{0.3cm}  \hfill (Energy release in MeV)
	
{ \small All these four branches are called pp chain reaction.}
\end{notebox}
