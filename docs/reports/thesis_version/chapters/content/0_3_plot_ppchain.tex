\begin{wrapfigure}{r}{0.5\textwidth}
	\centering
	\vspace{-0cm}
	\includegraphics[width = 0.6\textwidth]{../plots/0_images/ppchain}
	\caption{\small{Following proton-proton fusion, accumulation of Deuterium opens possibility for fusing with one more proton to form Helion - an isotope of Helium with two protons and one neutron. Rising temperature fuses two Helion atoms to form Helium (two protons and two neutrons) which is a stable element and emits two Hydrogen atoms. This series of reactions from Hydrogen to Helium is called p-p chain (pp I branch) reaction. Other rarer reactions can also occur inside the star. For instance, pp II branch when Helion fuses with Helium to form Beryllium (4 protons and 3 neutron). then either Beryllium decays to Lithium (3 protons and 4 neutron), Lithium fuses with proton and breakdown into two Helium atoms. In pp III branch, Beryllium fuses with proton to form Boron (4 proton and 4 neuton) which is unstable, breaking down into Helium. The most rare case is pp IV branch, when Helion captures a proton and form Helium. All these four branches are called pp chain reaction. Source: F. Reines, Ann. Rev. Nucl. Sci. 10 (1960) 1-26}}
	\label{ppchain}
\end{wrapfigure}
