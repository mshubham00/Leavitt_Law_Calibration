\documentclass[12pt,a4paper]{article}
% Language & font

\usepackage[a4paper,top=2cm,bottom=2cm,left=2cm,right=2cm,marginparwidth=1.75cm]{geometry}
\usepackage[british]{babel}
\usepackage[T1]{fontenc}
\usepackage{lmodern}
\usepackage{csquotes}
             
% Citations 
\usepackage[style=apa, backend=biber]{biblatex}
\addbibresource{references.bib}  % Your .bib file
\DeclareLanguageMapping{british}{british-apa}
 
% Add additional APA 7th edition requirements
\DeclareLanguageMapping{british}{british-apa} % Set language mapping
\DeclareFieldFormat[article]{volume}{\apanum{#1}} % Format volume number

% Modify 'and' to '&' in the bibliography
\renewcommand*{\finalnamedelim}{%
  \ifnumgreater{\value{liststop}}{2}{\finalandcomma}{}%
  \addspace\&\space}


  
\usepackage{graphicx, subcaption, booktabs, float, caption, xcolor, amsmath, amsfonts, mathrsfs, setspace, titlesec, lineno,algorithm, algorithmicx, algpseudocode, authblk}
\usepackage[table]{xcolor}  % you already have this
\onehalfspacing
\titleformat{\section}{\normalfont\Large\bfseries}{\thesection.}{1em}{}
\renewcommand\linenumberfont{\normalfont\scriptsize\sffamily\color{blue}}
\rightlinenumbers
\usepackage[colorlinks=true, allcolors=blue]{hyperref}
\makeatletter
\@ifundefined{pdfsuppresswarningpagegroup}{}{
  \pdfsuppresswarningpagegroup=1 
}  
\makeatother 
%-------------------------------------------
% Paper Head 
%-------------------------------------------
\title{COMPOSITE WESENHEIT BASED GALACTIC BVIJHK LEAVITT LAW CALIBRATION}

\author[1]{Shubham Mamgain}
\author[2]{Dr. Jesper Storm}
\author[1,2]{Prof. Dr. Maria Rosa Cioni}
\affil[1]{\small \textit{Institute of Physics and Astronomy, University of Potsdam}}
\affil[2]{\small \textit{Dwarf Galaxy and Galactic Halo, AIP Potsdam}}

\date{05.11.2025}  % Remove date 
   
\newcommand{\stars}{94}  % defines \mydistance to print μ
\newcommand{\dis}{p}
\newcommand{\filename}{94_jesper}
\newcommand{\ufilename}{94_jesper_JK}
  
\begin{document}
\maketitle

\begin{abstract}
From the definition of the total-to-selective absorption ratio, composite Wesenheit magnitudes are derived for $BVIJHK$ photometry of Galactic Cepheids. Fouque's extinction law ($A^V_\lambda$) adjusted to achieve near zero deviation between apparent and true wesenheit magnitudes for all the pair combinations of color indexes when $R^{BV}_V = 3.23$. The adjusted extinction law is used for determining $BVIJHK$ extinctions for 94 Galactic Cepheids with available Fernie reddenings and Gaia parallax. Metallicity effects are not considered in this study.
 \\
\textbf{Method:} Initially, $IJHK$ Period–Luminosity (PL) residuals are correlated with the corresponding Period–Wesenheit (PW) residuals to resolve the degeneracy between band-dependent extinction and distance uncertainties. A star-by-star adjustment of the distance modulus performed to minimize reddening dispersion across bands, yielding a first order distance - reddening correction pair for each individual Cepheid. $(J-K)$ based residual correlation provides an accurate estimation of modulus correction. After implementing the first - order corrections in the input data, the procedure is repeated for the $BVIJ$ bands to obtain precise reddening corrections using the Wesenheit color $(V - I)$. The final corrections are then applied to derive the $BVIJHK$ Leavitt Laws and verified with 17 Cluster Cepheids.
 \\
\textbf{Result:} Distance - Reddening - Luminosity calibrated V and K bands Galactic Leavitt Law:
\[
V_{\mathrm{Gaia}} (\sigma : 0.029) = (-2.268  \pm 0.01) \, ( \log P - 1 ) + \; (-3.953 \pm 0.003) 
\] 
\[
K_{\mathrm{Gaia}} (\sigma : 0.011) = (-2.971 \pm 0.004) \, ( \log P - 1 )  + \; (-5.660 \pm 0.001) 
\]
The distance moduli to the Magellanic Clouds, estimated from $VIJK$ photometry,  are found to be $18.365 \pm 0.110$~mag for the Large Magellanic Cloud (LMC)  and $19.068 \pm 0.038$~mag for the Small Magellanic Cloud (SMC).
\end{abstract}
\textbf{Keywords: } Interstellar Extinction Law: Wesenheit Magnitude: Galactic Cepheid: Leavitt Law: % must put \@keyword or \@wordskey or directly the text of the keywords
 
%-------------------------------------------
% Paper Body
%-------------------------------------------
%--- Section ---%

\begin{table}[h]
    \centering
    \small
    \renewcommand{\arraystretch}{1.5}
    \caption*{Symbols and their descriptions}
    \resizebox{\textwidth}{!}{%
    \begin{tabular}{ll|ll}
    \hline
    Symbol & Description & Symbol & Description \\ \hline
    $m_\lambda$ & Apparent Magnitude  & $\alpha_{\lambda}$ & Period Luminosity Slope \\
    $\mu$ & Distance Modulus & $\gamma_{\lambda}$ & Period Luminosity Intercept \\
    $M_\lambda$ & Absolute Magnitude  & $\Delta W^{12}_{\lambda}$ & Period Wesenheit Residuals \\
    $E_{BV}$ & Interstellar Reddening  & $\alpha^{12}_{\lambda}$ & Period Wesenheit Slope \\
    $A_\lambda$ & Interstellar Extinction & $\gamma^{12}_{\lambda}$ & Period Wesenheit Intercept \\
    $R_\lambda^{12}$ & Total-to-selective Absorption & $\rho^{12}_{\kappa \lambda}$ & Residual Correlation Slope  \\
    $M_\lambda^0$ & True Absolute Magnitude  & $\Delta^{12}_{\kappa \lambda}$ & PL-PW Correlation Residuals  \\
    $W_\lambda^{12}$ & Wesenheit Magnitude  & $\delta A^{12}_{\kappa \lambda}$ & Extinction Correction  \\
    $\Delta M_{\lambda}$ & Period Luminosity Residuals  & $\delta E^{12}_{\kappa \lambda}$ & Reddening Correction  \\ \hline
    \end{tabular}
    }
    \label{tab:symbols}
    \end{table}
 
%--- Section ---%
\section{Definitions}

\subsection{Distance, Reddening and Luminosity}
\sloppy
Luminosity represents the measured photon flux received within a specific wavelength band, here corresponding to the $BVIJHK$ bands. The light emitted by a star, characterized by its intrinsic absolute magnitude $M_\lambda^0$, travels over large distances $\mu$ and undergoes extinction $A_\lambda$ as it passes through the interstellar medium (ISM) before reaching the observer’s detector with the observed intensity $m_\lambda$. 
  
In this framework, the relation for the observed apparent magnitude can be expressed as:
\begin{align}
 m_\lambda = \mu + A_\lambda +  M_\lambda^0 
\end{align}
where $m_\lambda$ is the apparent magnitude in a given band $\lambda$, $M_\lambda^0$ is the intrinsic absolute magnitude, $\mu$ is the distance modulus, and $A_\lambda$ is the interstellar extinction in that band.

  
On the right-hand side, the first term represents the distance modulus, $\mu$. For computational convenience, distances measured in parsecs are converted into dimensionless logarithmic units, expressed in magnitudes. The distances for the 94 Galactic Cepheids in the dataset are determined using two independent methods: (i) the Infrared Surface Brightness (IRSB) method \textcite{storm2011}, and (ii) parallax measurements from \textit{Gaia} DR3 \textcite{GaiaDR32023}. In this report, calibration is done for Gaia distances.
   
\begin{align*}
	\mu & = 5 \log D [pc] - 5
\end{align*}

The interstellar medium (ISM) scatters, absorbs, and emits photons depending on the chemical composition, size, and abundance of its constituent particles, thereby imprinting characteristic signatures on the light spectrum at specific frequencies (or wavelengths). The reduction in light intensity within a given band due to these interactions with the ISM is referred to as \textit{interstellar extinction}.

\begin{align*}
	A_\lambda & = m_\lambda - m_\lambda^0
\end{align*}

The wavelength-dependent extinction for each individual Cepheid is determined from the measurement of the color excess, $E_{BV}$, along the line of sight \textcite{fernie1995}. Using the visual Galactic reddening ratio, $R_V = 3.23$ \textcite{sandage2004}, within the framework of the Galactic extinction law, $A_\lambda / A_V$ \textcite{fouque2007}, the interstellar extinction for the $BVIJHK$ bands can be estimated as follows:

\begin{align*}
A_\lambda & = \frac{A_\lambda}{A_V} \times R_V \times E_{BV} \\
\end{align*}

Here, the final factor, the color excess $E_{BV}$, quantifies the relative extinction difference between two photometric bands. In other words, it can also be expressed as the deviation of the observed color index $(B-V)$ from the intrinsic (or 'true') color index $(B-V)_0$:

\begin{align*}
E_{BV} & = (B-V) - (B-V)_0 \\
        & = (B-B_0) - (V-V_0) \\
        & = A_B - A_V \\
\end{align*}

The intermediate factor, the reddening ratio $R_V = A_V / E_{BV}$, represents the extinction-to-reddening ratio in the visual band. In the most general form, if the color excess between two bands is defined as $E_{12} = A_{m_1} - A_{m_2}$, then the corresponding reddening ratio can be expressed as:

\begin{align*}
R_\lambda^{12} = A_\lambda/E_{12}
\end{align*}

Color excess from one combination of bands $E_{12}$ can be transformed into any other bands combination as follows:

\begin{align*}
E_{12} & = A_{1} - A_{2} \\
    & = R_{1}^{BV} * E_{BV} - R_{2}^{BV} * E_{BV} \\
    & = (R_{1}^{BV} - R_{2}^{BV})*E_{BV} \\
\end{align*}


This leads the transformation law for reddening ratio R:

\begin{align*}
    R_\lambda^{12} = \frac{R_\lambda^{BV}}{R_1^{BV} - R_2^{BV}}
    \label{eq:Rlambda}
\end{align*}

The values of $R_\lambda^{BV}$ are calculated using extinction law as mentioned in Table 1.

\begin{align*}
R_\lambda^{BV} = \frac{A_\lambda}{A_V} \times R_V^{BV} 
\end{align*}

\subsection{Wesenheit Magnitude}

Since the reddening ratio, $R_\lambda^{12}$, quantifies the effect of the interstellar medium on the incoming light, it can be used to define a reddening-free magnitude corresponding to the star's true absolute magnitude \textcite{madore1982}. This relation can be expressed as:

\begin{align*}
R_\lambda^{12} & = A_\lambda / E_{12} \\
    & = \frac{m_\lambda - m_\lambda^0}{(m_1 - m_2)-(m_1 - m_2)_0} \\
\end{align*}

On rearranging the terms, one can get the Wesenheit function $W_\lambda^{12}$ corresponding to color $m_1 - m_2$.

\begin{align}
m_\lambda - R_\lambda^{12}(m_1 - m_2) & = m_\lambda^0 - R_\lambda^{12}(m_1 - m_2)_0 \\
W_\lambda^{12} & = W_0
\end{align}

This implies that, if $R$ is accurately known, the apparent Wesenheit magnitude and the absolute Wesenheit magnitude should be identical. Here, we test the validity of this claim by comparing the deviations between the two versions of the Wesenheit magnitudes, where $R$ is derived from the Fouque extinction law as given in Table 1.

	\begin{figure}[h]
		\centering
		\begin{subfigure}[b]{\textwidth}
			\centering
            \includegraphics[width=\textwidth]{./plots/1_datacleaning/FB.pdf}
		\end{subfigure}
		\begin{subfigure}[b]{\textwidth}
			\centering
            \includegraphics[width=\textwidth]{./plots/1_datacleaning/FV.pdf}
		\end{subfigure}
		\begin{subfigure}[b]{\textwidth}
			\centering
            \includegraphics[width=\textwidth]{./plots/1_datacleaning/FI.pdf}
		\end{subfigure}
        \caption{For $BVI$ photometry, the deviation between apparent Wesenheit magnitude and absolute Wesenheit magnitude for all the color combination indicates a potential error in the reddening law.}
	\end{figure}

This deviation in wesenheit $W-W_0$ would be zero, when following equality would be satisfying:

\begin{align}
    \delta A_\lambda - R_\lambda \delta E_{BV} - \delta R E_{BV} = 0
    \label{eq:wes_dev}
\end{align}
 
Considering the last term of the above equation, \textcite{fouque2007} extinction law have been adjusted to achieve minimum deviation in the wesenheit magnitudes.

	\begin{figure}[h]
		\centering
        \begin{subfigure}[b]{\textwidth}
		\centering
            \includegraphics[width=\textwidth]{./plots/1_datacleaning/SB.pdf}
		\end{subfigure}		
		\begin{subfigure}[b]{\textwidth}
			\centering
            \includegraphics[width=\textwidth]{./plots/1_datacleaning/SV.pdf}
		\end{subfigure}
		\quad
		\begin{subfigure}[b]{\textwidth}
		\centering
            \includegraphics[width=\textwidth]{./plots/1_datacleaning/SI.pdf}
		\end{subfigure}
        \caption{With the adjusted Fouqué extinction law, the deviation between the apparent and absolute Wesenheit magnitudes is reduced by a factor of ten.}
	\end{figure}


\begin{table}[ht]
\centering
\caption{Reddening Ratio metrics for composite photometry.}
\label{tab:reddening}
\resizebox{\textwidth}{!}{%
\begin{tabular}{| l r | l r | l r | l r | l r | l r |}
\hline
\multicolumn{12}{|c|}{\textbf{Extinction Law \textcite{fouque2007}}} \\ % <-- added \\ and |l| for border alignment
\hline
\rowcolor{yellow!20}
\textbf{$A_B^{V}$} & 1.31 &
\textbf{$A_V^{V}$} & 1.0 &
\textbf{$A_I^{V}$} & 0.608 &
\textbf{$A_J^{V}$} & 0.292 &
\textbf{$A_H^{V}$} & 0.181 &
\textbf{$A_K^{V}$} & 0.119 \\
\hline
\multicolumn{12}{|c|}{\textbf{Adjusted Extinction Law}} \\ % <-- added \\ and |l| for border alignment
\hline
\rowcolor{yellow!20}
\textbf{$A_B^{V}$} & 1.2574 &
\textbf{$A_V^{V}$} & 1.0 &
\textbf{$A_I^{V}$} & 0.609 &
\textbf{$A_J^{V}$} & 0.2967 &
\textbf{$A_H^{V}$} & 0.1816 &
\textbf{$A_K^{V}$} & 0.1231 \\
\hline
\multicolumn{12}{|c|}{\textbf{Reddening Ratio calculated using $R_V^{BV}$ = 3.23}} \\
\hline
\rowcolor{gray!20}
\textbf{$R_B^{12}$} &  &
\textbf{$R_V^{12}$} &  &
\textbf{$R_I^{12}$} &  &
\textbf{$R_J^{12}$} &  &
\textbf{$R_H^{12}$} &  &
\textbf{$R_K^{12}$} &  \\
\hline
\hline
\rowcolor{green!25}
BBV & 4.887 & VBV & 3.887 & IBV & 2.367 & JBV & 1.153 & HBV & 0.706 & KBV & 0.478 \\
BBI & 1.94 & VBI & 1.543 & IBI & 0.939 & JBI & 0.458 & HBI & 0.28 & KBI & 0.19 \\
BBJ & 1.309 & VBJ & 1.041 & IBJ & 0.634 & JBJ & 0.309 & HBJ & 0.189 & KBJ & 0.128 \\
BBH & 1.169 & VBH & 0.93 & IBH & 0.566 & JBH & 0.276 & HBH & 0.169 & KBH & 0.114 \\
\rowcolor{yellow!25}
BBK & 1.108 & VBK & 0.882 & IBK & 0.537 & JBK & 0.262 & HBK & 0.16 & KBK & 0.108 \\
BVI & 3.216 & VVI & 2.557 & IVI & 1.557 & JVI & 0.759 & HVI & 0.464 & KVI & 0.315 \\
BVJ & 1.788 & VVJ & 1.422 & IVJ & 0.866 & JVJ & 0.422 & HVJ & 0.258 & KVJ & 0.175 \\
BVH & 1.537 & VVH & 1.222 & IVH & 0.744 & JVH & 0.363 & HVH & 0.222 & KVH & 0.15 \\
BVK & 1.434 & VVK & 1.14 & IVK & 0.694 & JVK & 0.338 & HVK & 0.207 & KVK & 0.14 \\
BIJ & 4.025 & VIJ & 3.201 & IIJ & 1.95 & JIJ & 0.95 & HIJ & 0.581 & KIJ & 0.394 \\
BIH & 2.943 & VIH & 2.341 & IIH & 1.425 & JIH & 0.694 & HIH & 0.425 & KIH & 0.288 \\
BIK & 2.587 & VIK & 2.057 & IIK & 1.253 & JIK & 0.61 & HIK & 0.374 & KIK & 0.253 \\
\rowcolor{red!25} 
BJH & 10.947 & VJH & 8.706 & IJH & 5.302 & JJH & 2.583 & HJH & 1.581 & KJH & 1.071 \\
BJK & 7.24 & VJK & 5.758 & IJK & 3.506 & JJK & 1.708 & HJK & 1.046 & KJK & 0.708 \\
\rowcolor{red!25}
BHK & 21.376 & VHK & 17.0 & IHK & 10.353 & JHK & 5.044 & HHK & 3.087 & KHK & 2.091 \\
\hline
\multicolumn{12}{|c|}{\textbf{LMC Extinction Law  with $R_V = 3.4$ \textcite{wang2023}}} \\ % <-- added \\ and |l| for border alignment
\hline 
\rowcolor{yellow!20}
\textbf{$A_B^{V}$} & 1.32 &
\textbf{$A_V^{V}$} & 1.0 &
\textbf{$A_I^{V}$} & 0.65 &
\textbf{$A_J^{V}$} & 0.30 &
\textbf{$A_H^{V}$} & 0.20 &
\textbf{$A_K^{V}$} & 0.15 \\
\hline
\multicolumn{12}{|c|}{\textbf{SMC Extinction Law with $R_V = 2.53$}} \\ % <-- added \\ and |l| for border alignment
\hline
\rowcolor{yellow!20}
\textbf{$A_B^{V}$} & 1.40 &
\textbf{$A_V^{V}$} & 1.0 &
\textbf{$A_I^{V}$} & 0.7 &
\textbf{$A_J^{V}$} & 0.38 &
\textbf{$A_H^{V}$} & 0.28 &
\textbf{$A_K^{V}$} & 0.2 \\
\hline
\end{tabular}% 
}
\end{table}

	\begin{figure}[h]
		\centering
		\begin{subfigure}[b]{0.49\textwidth}
			\centering
            \includegraphics[width=\textwidth]{./plots/1_datacleaning/Rratio1.pdf}
		\end{subfigure}
		\begin{subfigure}[b]{0.49\textwidth}
			\centering
            \includegraphics[width=\textwidth]{./plots/1_datacleaning/Rratio2.pdf}
		\end{subfigure}
        \caption{Variation in reddening ratio for different color index}
	\end{figure}

After fine-tuning the Fouqué extinction law for our dataset, $R_\lambda$ is assumed to be accurately known to the required precision, eliminating the last term in the above equation.

\subsection{Error Contribution in Absolute and Wesenheit magnitude}


Following are the definitions of 'true' absolute magnitude and wesenheit magnitude for color (B-V): 

\begin{align}
    M_\lambda^0 & = m_\lambda - R_\lambda^{BV}*E(B-V) - \mu\\
    W_\lambda^{BV} & = m_\lambda - R_\lambda^{BV} * (B-V) - \mu
\end{align}

Differentiating above equations. 

\begin{align*}
    \delta M_\lambda^0 & = \delta m_\lambda - \delta R_\lambda^{BV}*E(B-V) - R_\lambda^{BV}* \delta E(B-V) - \delta \mu\\
    \delta W_\lambda^{BV} & = \delta m_\lambda - \delta R_\lambda^{BV} * (B-V) - R_\lambda^{BV} * \delta (B-V)  - \delta \mu
\end{align*}


Assume that the apparent luminosities are measured precisely ($\delta m_\lambda \to 0$) and that the adjusted extinction law is accurate ($\delta R \to 0$). Even so, the distance modulus ($\delta \mu$) and the color excess ($\delta E_{BV}$) may still contain errors for individual Cepheids. Under these conditions, it follows that:

\begin{align}
    \delta M_\lambda & = - ( R_\lambda^{BV}* \delta E(B-V) + \delta \mu)\\
    \delta W_\lambda^{BV} & = - \delta \mu
\end{align}

This pair of equations indicates that both the Wesenheit magnitude and the absolute magnitude are sensitive to errors in distance. However, unlike the absolute magnitude, the Wesenheit magnitude is insensitive to errors in reddening. This key property of the Wesenheit magnitude is particularly useful for decoupling the error budget between distance and reddening when applied to Cepheids.

\section{Galactic Cepheids: Leavitt Law }

Classical Cepheids exhibit a linear relationship between their pulsation period ($\log P$) and their true luminosity ($M_\lambda^0$), which can be modeled as:

\begin{align}
\bar{M}_\lambda = \alpha_\lambda (logP) + \gamma_\lambda
\end{align}

	\begin{figure}[h]
		\centering
        \begin{subfigure}[b]{\textwidth}
		\centering
            \includegraphics[width=\textwidth]{./plots/2_PLPW/\filename_0BVIJHK_\dis.pdf}
		\end{subfigure}		
        \caption{BVIJHK Leavitt Law prior to calibration}
	\end{figure}


This linear regression equation provides the absolute magnitude ($\bar{M}_\lambda$) of any Cepheid from its pulsation period. Our dataset includes $BVIJHK$ photometry of Galactic Cepheids, resulting in six Leavitt Laws, each modeled by two parameters: the slope ($\alpha_\lambda$) and the intercept ($\gamma_\lambda$).

The deviation of the observed luminosity ($M_\lambda^0$) from the modeled value ($\bar{M}_\lambda$) is denoted as $\Delta M_\lambda$.

\begin{align*}
    \Delta M_\lambda = M_\lambda^0 - \bar{M}_\lambda
\end{align*}


	\begin{figure}[h]
		\centering
        \begin{subfigure}[b]{\textwidth}
		\centering
            \includegraphics[width=\textwidth]{./plots/2_PLPW/\filename_BVIJHK_0_\dis_residuals.pdf}
		\end{subfigure}		
        \caption{Residuals of the $BVIJHK$ Leavitt Laws show a decreasing scatter with increasing wavelength, indicating a dependence on interstellar extinction.}
	\end{figure}


The Wesenheit magnitude - based Leavitt Law can be derived using a reference color index ($m_1 - m_2$). For each combination of band and color index, the reddening-free form of the Leavitt Law can be modeled as:

\begin{align}
\bar{W}_\lambda^{12} = \alpha_\lambda^{12} (logP) + \gamma_\lambda^{12}
\end{align}

	\begin{figure}[h]
		\centering
        \begin{subfigure}[b]{\textwidth}
		\centering
            \includegraphics[width=\textwidth]{./plots/2_PLPW/\filename_JK_BVIJHK_\dis.pdf}
		\end{subfigure}		
        \caption{BVIJHK Period-Wesenheit Relation for (J-K) color.}
	\end{figure}


For $BVIJHK$ photometry, there are 15 possible combinations of color indices. Considering six observed bands with each color index results in 90 composite Wesenheit magnitudes ($W_\lambda^{12}$). The deviation of the Wesenheit magnitude from the corresponding Period-Wesenheit (PW) relation is given by:


	\begin{figure}[h]
		\centering
        \begin{subfigure}[b]{\textwidth}
		\centering
            \includegraphics[width=\textwidth]{./plots/2_PLPW/\filename_BVIJHK_JK_\dis_residuals.pdf}
		\end{subfigure}		
        \caption{Residuals of the $BVIJHK$ Wesenheit-Leavitt Law for the $(J-K)$ color show an approximately constant scatter across all bands, indicating that the Wesenheit magnitude is insensitive to interstellar extinction. Therefore, scatter is attributed primarily to errors in the distance modulus.}
	\end{figure}


\begin{align*}
    \Delta W_\lambda^{12} = W_\lambda^{12} - \bar{W}_\lambda^{12}
\end{align*}



	\begin{figure}[h]
		\centering
        \begin{subfigure}[b]{\textwidth}
		\centering
            \includegraphics[width=\textwidth]{./plots/2_PLPW/\stars_BVIJHK_VI_\dis_residuals.pdf}
		\end{subfigure}		
        \caption{Residuals of BVIJHK Wesenheit-Leavitt Law for (V-I) color.}
	\end{figure}


\begin{table}[h!]
\centering
\caption{Slope and Intercept of BVIJHK Leavitt Law using 94 Galactic Cepheids}
\label{tab:reddening}
\resizebox{\textwidth}{!}{%
\begin{tabular}{|c|c|c|c|c|c|c|}
\hline
\rowcolor{gray!20}
 & B (err) & V (err) & I (err) & J (err) & H (err) & K (err) \\
\hline
\hline
\multicolumn{7}{|c|}{\textbf{Slope and Intercept of Period-Luminosity Relation}} \\ 
\hline
Slope & -1.83 (0.159) & -2.26 (0.131) & -2.57 (0.105) & -2.79 (0.091) & -2.92 (0.086) & -2.97 (0.086) \\
Intercept & -4.45 (0.048) & -5.00 (0.039) & -5.37 (0.031) & -5.53 (0.027) & -5.79 (0.026) & -5.79 (0.026) \\
\hline
\hline
\multicolumn{7}{|c|}{\textbf{Slope of Period-Wesenheit Relations}} \\
\hline
BV & -3.92 (0.132) & -3.92 (0.132) & -3.57 (0.113) & -3.28 (0.095) & -3.22 (0.089) & -3.18 (0.087) \\
BI & -3.25 (0.101) & -3.39 (0.105) & -3.25 (0.100) & -3.12 (0.091) & -3.12 (0.087) & -3.11 (0.086) \\
BJ & -3.08 (0.089) & -3.25 (0.094) & -3.17 (0.094) & -3.08 (0.089) & -3.10 (0.086) & -3.10 (0.085) \\
BH & -3.10 (0.085) & -3.27 (0.091) & -3.18 (0.092) & -3.09 (0.088) & -3.10 (0.085) & -3.10 (0.085) \\
BK & -3.10 (0.085) & -3.27 (0.090) & -3.18 (0.091) & -3.09 (0.088) & -3.10 (0.085) & -3.10 (0.085) \\
\rowcolor{yellow!20}
VI & -2.82 (0.097) & -3.04 (0.097) & -3.04 (0.097) & -3.02 (0.090) & -3.06 (0.086) & -3.07 (0.085) \\
VJ & -2.78 (0.088) & -3.01 (0.088) & -3.02 (0.090) & -3.01 (0.088) & -3.05 (0.085) & -3.07 (0.085) \\
VH & -2.84 (0.083) & -3.06 (0.084) & -3.05 (0.088) & -3.03 (0.087) & -3.06 (0.084) & -3.07 (0.084) \\
VK & -2.86 (0.084) & -3.07 (0.084) & -3.06 (0.088) & -3.03 (0.087) & -3.07 (0.084) & -3.07 (0.084) \\
IJ & -2.72 (0.102) & -2.97 (0.094) & -3.00 (0.088) & -3.00 (0.088) & -3.05 (0.085) & -3.06 (0.085) \\
IH & -2.87 (0.089) & -3.09 (0.085) & -3.07 (0.083) & -3.03 (0.086) & -3.07 (0.083) & -3.08 (0.084) \\
IK & -2.89 (0.090) & -3.10 (0.085) & -3.08 (0.084) & -3.04 (0.086) & -3.07 (0.083) & -3.08 (0.084) \\
\rowcolor{red!20}
JH & -3.27 (0.081) & -3.40 (0.079) & -3.26 (0.079) & -3.13 (0.081) & -3.13 (0.081) & -3.11 (0.082) \\
JK & -3.19 (0.088) & -3.34 (0.082) & -3.22 (0.081) & -3.11 (0.083) & -3.11 (0.082) & -3.11 (0.083) \\
\rowcolor{red!20}
HK & -3.04 (0.159) & -3.22 (0.131) & -3.15 (0.105) & -3.07 (0.091) & -3.09 (0.086) & -3.09 (0.086) \\

\hline
\multicolumn{7}{|c|}{\textbf{Intercept of Period-Wesenheit Relations}} \\
\hline
BV & -7.11 (0.039) & -7.11 (0.039) & -6.66 (0.034) & -6.16 (0.029) & -6.17 (0.027) & -6.05 (0.026) \\
BI & -6.24 (0.030) & -6.42 (0.031) & -6.24 (0.030) & -5.96 (0.027) & -6.05 (0.026) & -5.96 (0.026) \\
BJ & -5.87 (0.027) & -6.12 (0.028) & -6.06 (0.028) & -5.87 (0.027) & -5.99 (0.026) & -5.93 (0.025) \\
BH & -6.02 (0.025) & -6.24 (0.027) & -6.13 (0.027) & -5.90 (0.026) & -6.02 (0.025) & -5.94 (0.025) \\
BK & -5.94 (0.025) & -6.18 (0.027) & -6.09 (0.027) & -5.89 (0.026) & -6.00 (0.025) & -5.94 (0.025) \\
\rowcolor{yellow!20}
VI & -5.66 (0.029) & -5.96 (0.029) & -5.96 (0.029) & -5.82 (0.027) & -5.96 (0.026) & -5.91 (0.025) \\
VJ & -5.42 (0.026) & -5.76 (0.026) & -5.84 (0.027) & -5.76 (0.026) & -5.93 (0.025) & -5.88 (0.025) \\
VH & -5.67 (0.025) & -5.96 (0.025) & -5.96 (0.026) & -5.82 (0.026) & -5.96 (0.025) & -5.91 (0.025) \\
VK & -5.59 (0.025) & -5.90 (0.025) & -5.93 (0.026) & -5.80 (0.026) & -5.95 (0.025) & -5.90 (0.025) \\
IJ & -5.10 (0.031) & -5.51 (0.028) & -5.69 (0.026) & -5.69 (0.026) & -5.88 (0.025) & -5.85 (0.025) \\
IH & -5.68 (0.027) & -5.97 (0.025) & -5.97 (0.025) & -5.82 (0.026) & -5.97 (0.025) & -5.91 (0.025) \\
IK & -5.53 (0.027) & -5.85 (0.025) & -5.89 (0.025) & -5.79 (0.026) & -5.94 (0.025) & -5.89 (0.025) \\
\rowcolor{red!20}
JH & -7.23 (0.024) & -7.21 (0.024) & -6.72 (0.024) & -6.19 (0.024) & -6.19 (0.024) & -6.06 (0.024) \\
JK & -6.29 (0.026) & -6.46 (0.024) & -6.27 (0.024) & -5.97 (0.025) & -6.05 (0.024) & -5.97 (0.025) \\
\rowcolor{red!20}
HK & -4.45 (0.048) & -5.00 (0.039) & -5.37 (0.031) & -5.53 (0.027) & -5.79 (0.026) & -5.79 (0.026) \\
\hline
\end{tabular}%
}
\end{table}

Correlation between pulsation period and luminosity (true absolute and wesenheit) of Cepheids produces $6 + 90$ linear equations, each characterized by a slope and intercept, along with the residuals for each Cepheid. A summary is provided below in Table ~\ref{tab:reddening}

\begin{center}
    PL relation: $\alpha_\lambda$ | $\gamma_\lambda$ | $\Delta M_\lambda $ |  $\implies$  $6$ equations \\
    PW relation: $\alpha_\lambda^{12}$ | $\gamma_\lambda^{12} $  | $\Delta W_\lambda^{12}$ $\implies$  $90$ equations
\end{center} 



%--- Section ---%
\subsection{PL-PW Residuals}

A crucial difference between the two types of relations is that the residuals of the Period-Wesenheit (PW) relation are free from reddening errors, whereas those of the Period-Luminosity (PL) relation are not. The correlation between the PL and PW residuals ($\Delta_{\rm PL} - \Delta_{\rm PW}$ correlation) provides quantitative insights into the error budget associated with reddening and distance modulus. To develop a mathematical framework, two cases must be considered:

\textbf{i) When there is error only in distance, ($\delta \mu \neq 0$, $\delta E_{12}= 0$):} 
\begin{center}
Deviation in PL : $\Delta M_\lambda = \delta \mu$ \\
Deviation in PW : $\Delta W_\lambda^{12} = \delta \mu$
\end{center}

In the residual correlation plots, \textit{an error in distance shifts the star equally along both axes from its original position}, meaning that Cepheid with minimal or no reddening error would lie perfectly along a line with slope 1. This behavior is clearly evident in the infrared bands, where reddening errors are expected to be minimal at longer wavelengths.


\textbf{ii) When there is error in reddening alone, ($\delta E_{12} \neq 0$, $\delta \mu = 0$): }
\begin{center}
Deviation in PL : $\Delta M_\lambda = R^{12}_\lambda \delta E_{12}$ \\
Deviation in PW : $\Delta W_\kappa^{12}$ = 0
\end{center}

In this case, only the PL residuals are affected, whereas the PW residuals remain unchanged, since the Wesenheit magnitude is reddening - independent by definition. This means that the star shifts along the PL residual axis but not along the PW residual axis. Consequently, \textit{a vertical displacement of the star in $\Delta$–$\Delta$ plots originates from the reddening error}.


\section{Distance - Reddening Calibration Methodology}

To decouple the reddening and distance errors for individual Cepheids, the residuals of the PL relations ($\Delta M_\lambda$) are plotted against the corresponding PW residuals ($\Delta W_\kappa^{12}$). Fitting a regression line with slope $\rho_{\kappa \lambda}^{12}$ and zero intercept yields the residuals for each band and choice of Wesenheit color. The resulting $\Delta–\Delta$ residuals are denoted as:

\begin{align*}
       \Delta_{\kappa \lambda}^{12} = \Delta M_\lambda - \rho_{\kappa \lambda}^{12} \times \Delta W_\kappa^{12}
\end{align*}

This vertical deviation from the regression line indicates the extinction correction for the corresponding band, independent of any correction to the distance modulus.

\begin{align*}
       \delta A_{\kappa \lambda}^{12} (0) = \Delta_{\kappa \lambda}^{12} 
\end{align*}

By normalizing with $R_\lambda^{12}$, the extinction corrections ($\delta A^{12}_{\kappa \lambda}$) are converted into reddening corrections ($\delta E_{\kappa \lambda}^{12}$) as follows:

\begin{align*}
\delta E^{12}_{\kappa \lambda} (0) = \frac{\delta A^{12}_{\kappa \lambda }}{R_\lambda^{12}}
\end{align*}

If the distances of individual targets are precisely known, then reddening is the sole source of error. This implies that the corrections indicated by each band should be consistent with one another.

\begin{align*}
    \delta E^{12}_{\kappa B } = \delta E^{12}_{\kappa V} = ... = \delta E^{12}_{\kappa K} = \delta E^*_{12}
\end{align*}

If distances are accurately known, the reddening corrections derived from different bands would show no dispersion, as discussed above. In contrast, an incorrect distance for a given Cepheid would result in discrepancies among the reddening corrections estimated from the various bands.

To estimate the distance error, a range of possible distance corrections ($\delta \mu^i$) is considered. For each assumed correction, the reddening corrections for all bands are computed. The particular distance correction for which the reddening corrections are most consistent across all bands (i.e., exhibiting minimal variation) is adopted as the optimal distance – reddening correction pair for the individual Cepheid.

The variation in extinction ($\delta A^{12}_{\kappa \lambda}$) resulting from a distance modulus correction is calculated as follows:
 {\begin{align*}
     \delta A_{\kappa \lambda}^{12} (\delta \mu) & = (\Delta M_\lambda + \delta \mu) - \rho_{\kappa \lambda}^{12} \times (\Delta W_\kappa^{12} + \delta \mu) \\
     & = \delta A_{\kappa \lambda}^{12}(0) + \delta \mu( 1 - \rho_{\kappa \lambda}^{12} )
 \end{align*}}

The corresponding reddening corrections are then given by:


 \begin{align}
     \delta E_{\kappa \lambda}^{12} (\delta \mu) = \delta E_{\kappa \lambda}^{12} (0) + \frac{\delta \mu( 1 - \rho_{\kappa \lambda}^{12} )}{R_{\lambda}^{12}}
 \end{align}

For the optimal distance - reddening correction pair, the dispersion in the estimated reddening corrections from the BVIJHK bands is minimized.

\begin{align*}
    RMS(\delta E (\delta \mu))_\lambda = \frac{1}{6}\sum_{\lambda} (\delta E_{\kappa \lambda}^{12} (\delta \mu) - <\delta E_{\kappa \lambda}^{12} (\delta \mu)>_\lambda)^2
\end{align*}

\begin{align*}
    min(RMS(\delta E (\delta \mu))_\lambda)_\mu \implies (\delta \mu^*, \delta E^*_{12})
\end{align*}

These corrections are then applied to the original data as follows:

\begin{align}
    M_\lambda^* = M_\lambda^0 + \delta A_\lambda^* + \delta \mu^*
\end{align}

\subsection{Madore Approach}

In his analysis, \textcite{madore2017} used $W_V^{VI}$ as a reference for the reddening - free magnitude. He correlated the PW residuals ($\Delta W_V^{VI}$) with the BVRIJHK PL residuals ($\Delta M_\lambda$), yielding the corresponding slopes ($\rho_{V \lambda}^{VI}$) and residuals ($\Delta^{VI}_{V \lambda}$).

Let $\delta \mu^*$ denote the correction required in the distance modulus. Then, the reddening corrections $\delta E_{V \lambda}^{VI}$, as derived by Madore (2017), are given by:

\begin{align*}
    \delta E_{V \lambda}^{VI} (\delta \mu^*) & = \frac{\delta A_{V \lambda}^{VI} (\delta \mu^*)}{R_\lambda^{VI}}\\
 \end{align*}
 
 Converting $E_{VI}$ into $E_{BV}$
 \begin{align*}
     \delta E_{V \lambda}^{BV}(\delta \mu^*) & = \frac{\delta A_{V \lambda}^{VI} (\delta \mu^*)}{({R^{BV}_V - R^{BV}_I})R_\lambda^{VI}} \\
     & = \frac{\delta A_{V \lambda}^{VI}(\delta \mu^*)}{R^{BV}_\lambda}  
 \end{align*}

 
	\begin{figure}[htp]
		\centering
        \begin{subfigure}[b]{\textwidth}
		\centering
	\includegraphics[width=\linewidth]{./plots/3_deldel/\stars_deldel_M_0VI_\dis.pdf} 
    	\caption{\small Residual correlation according to Madore (2017), where the $y$-axis represents the residuals of the PL relation ($\Delta M_\lambda$) and the $x$-axis, fixed for all six bands, represents the residuals of the PW relation ($\Delta W_V^{VI}$).}
    \end{subfigure}	        
    \begin{subfigure}[b]{\textwidth}
		\centering
	\includegraphics[width=\linewidth]{./plots/3_deldel/\filename_deldel_S_0VI_\dis.pdf} 
    \caption{\small Residual correlation where both the axis varies with bands.}
    \end{subfigure}		
        \caption{Comparison of the slopes between the two cases shows that the slope of the regression line in the bottom panel approaches 1 more rapidly with increasing wavelength.}
	\end{figure}


By implementing these reddening and distance error corrections in his dataset of 59 Cepheids, he derived the Leavitt Laws as given in Table ~\ref{madorePLW}:

\begin{table}[ht]
\centering
\caption{Madore (2017): Slope and Intercept of Galactic BVIJHK Leavitt Law }
\label{madorePLW}
\begin{tabular}{|c|c|c|c|c|}
\hline
\rowcolor{gray!20}
Band & Slope  & $\sigma $ (mag)  & Zero  & $\sigma$(mag)\\
\hline
B & −2.277 & 0.121 & −3.214 & 0.036 \\
V& −2.670 & 0.095 & −3.944  & 0.028   \\
R& −2.874 & 0.075 & −4.396  & 0.022  \\
I& −2.983 & 0.064 & −4.706  & 0.019  \\
J& −3.198 &  0.054 & −5.258 &  0.016 \\
H& −3.333 & 0.055 & −5.558  & 0.017  \\
K& −3.377 & 0.055 & −5.659  & 0.016  \\
\hline
W, VI & −3.476 & 0.037 & −5.889 &  0.011  \\
W, BI & −3.600 & 0.033 & −5.997 &  0.010  \\
\hline
\end{tabular}%
\end{table}



\subsection{My Approach}

Instead of keeping $\Delta W_V^{VI}$ fixed, I have used $\Delta W_\lambda^{VI}$ — the reddening free magnitude for the respective band $m_\lambda$. This choice follows directly from the definition of the Wesenheit function: the reddening - free magnitude for a given band must be derived using $R_\lambda^{12}$, not $R_V^{12}$. Specifically, $R_V^{VI}$ gives the wesenheit magnitude corresponding to the $V$ band only for the $(V-I)$ color index. For the $B$ band, $R_B^{VI}(V-I)$ must be subtracted from $B$ to obtain the corresponding Wesenheit magnitude. Consequently, the PW residuals along x-axis in my approach will be varying with the respective PL residuals $\Delta M_\lambda$. The difference between Madore's and my PW residuals is represented by the operator $\delta_V^\lambda$ and is calculated as follows:
  

\begin{align*}
    \delta_V^\lambda \Delta W_\lambda^{VI} & = \Delta W_\lambda^{VI} - \Delta W_V^{VI}\\ 
    & = (W_\lambda^{VI} - \bar{W}_\lambda^{VI}) - (W_V^{VI} - \bar{W}_V^{VI}) \\
    & = (W_\lambda^{VI} - W_V^{VI}) - (\bar{W}_\lambda^{VI} - \bar{W}_V^{VI}) \\
    & = ((M_\lambda - M_V) - (R_\lambda^{VI}-R_V^{VI})(V-I)) - ((\alpha_\lambda^{VI}-\alpha_V^{VI}) \log P + (\gamma_\lambda^{VI} - \gamma_V^{VI})) \\
\end{align*}

This deviation in the PW residuals affects the slope of the regression line, $\rho$, which implies that the correlation residuals, $\Delta_{\kappa \lambda}^{12}$, will also be affected. In Madore's approach, the $\Delta-\Delta$ correlation residuals are expressed as $\Delta_{V \lambda}^{VI}$, since he correlated $\Delta M_\lambda$ with $\Delta W_V^{VI}$. In contrast, I have correlated $\Delta M_\lambda$ with $\Delta W_\lambda^{VI}$, so that the resulting residuals take the form $\Delta_{\lambda \lambda}^{VI}$.

Difference between both kinds of correlation residuals is:

\begin{align*}
    \delta_{V \lambda}^{\lambda \lambda} \Delta_{\lambda \lambda}^{VI} & = \Delta_{\lambda \lambda}^{VI} - \Delta_{V \lambda}^{VI} \\
    & = \delta A^S_\lambda (0) - A^M_\lambda(0) \\
    & = (\Delta M_\lambda - \rho_{\lambda \lambda}^{VI} \times \Delta W_\lambda^{VI}) - (\Delta M_\lambda - \rho_{V \lambda}^{VI} \times \Delta W_V^{VI}) \\
    & = \rho_{V \lambda}^{VI} \times \Delta W_V^{VI} - \rho_{\lambda \lambda}^{VI} \times \Delta W_\lambda^{VI} \\
\end{align*}


\subsection{Physical Significance of Residual-Slope}
The slope of the PL - PW residual correlations quantifies the contributions error sources. In this study, the absence of extinction errors, the distance remains the only source of error, which affects both axes equally, resulting in a slope of 1.

\begin{table}[h]
	\centering
    
    \caption{Variation of the $\Delta-\Delta$ correlation slope, $\rho_{\kappa \lambda}^{12}$, for fifteen Wesenheit colors, comparing Madore's approach (white) with my approach (yellow). The $(V-I)$ case is highlighted in red. The continuous decrease in slope-error with increasing wavelength, aligning along a slope of 1, is a strong indication of larger distance error with respect .}
    \label{tab:rho}
    \resizebox{\textwidth}{!}{%
	\begin{tabular}{|>{\columncolor{gray!20}}c|c|c|c|c|c|c|}
		\hline
    \rowcolor{gray!20}
  	$\rho$ & B (err) & V (err) & I (err) & J (err) & H (err) & K (err) \\
\hline
BV &  0.243 ( 0.084) &  0.398 ( 0.067) &  0.471 ( 0.056) &  0.479 ( 0.050) &  0.478 ( 0.045) &  0.471 ( 0.046) \\
	 \rowcolor{yellow!20}
	    &  0.243 ( 0.084) &  0.398 ( 0.067) &  0.653 ( 0.054) &  0.870 ( 0.035) &  0.929 ( 0.024) &  0.962 ( 0.017) \\
	 \hline
	 BI &  0.652 ( 0.092) &  0.758 ( 0.066) &  0.831 ( 0.045) &  0.798 ( 0.040) &  0.773 ( 0.035) &  0.765 ( 0.036) \\
	 \rowcolor{yellow!20}
	    &  0.652 ( 0.092) &  0.683 ( 0.068) &  0.832 ( 0.045) &  0.944 ( 0.025) &  0.968 ( 0.016) &  0.982 ( 0.011) \\
	 \hline
	 BJ &  0.881 ( 0.090) &  0.945 ( 0.060) &  0.975 ( 0.041) &  0.972 ( 0.021) &  0.938 ( 0.014) &  0.933 ( 0.016) \\
	 \rowcolor{yellow!20}
	    &  0.881 ( 0.090) &  0.835 ( 0.066) &  0.920 ( 0.041) &  0.972 ( 0.021) &  0.981 ( 0.014) &  0.990 ( 0.009) \\
	 \hline
	 BH &  0.917 ( 0.096) &  0.983 ( 0.065) &  1.015 ( 0.045) &  1.017 ( 0.025) &  0.988 ( 0.014) &  0.984 ( 0.014) \\
	 \rowcolor{yellow!20}
	    &  0.917 ( 0.096) &  0.859 ( 0.070) &  0.941 ( 0.043) &  0.983 ( 0.022) &  0.988 ( 0.014) &  0.995 ( 0.009) \\
	 \hline
	 BK &  0.945 ( 0.094) &  0.996 ( 0.064) &  1.022 ( 0.045) &  1.023 ( 0.024) &  0.992 ( 0.014) &  0.994 ( 0.009) \\
	 \rowcolor{yellow!20}
	    &  0.945 ( 0.094) &  0.880 ( 0.069) &  0.951 ( 0.042) &  0.986 ( 0.021) &  0.990 ( 0.013) &  0.994 ( 0.009) \\
	 \hline
	 \rowcolor{red!20}
     VI &  0.787 ( 0.086) &  0.847 ( 0.060) &  0.907 ( 0.037) &  0.854 ( 0.035) &  0.820 ( 0.032) &  0.813 ( 0.034) \\
	 \rowcolor{red!20}
	    &  0.870 ( 0.077) &  0.847 ( 0.060) &  0.907 ( 0.037) &  0.974 ( 0.020) &  0.984 ( 0.013) &  0.991 ( 0.009) \\
	 \hline
	 VJ &  0.938 ( 0.087) &  0.980 ( 0.057) &  1.000 ( 0.039) &  0.994 ( 0.017) &  0.956 ( 0.010) &  0.953 ( 0.011) \\
	 \rowcolor{yellow!20}
	    &  1.053 ( 0.072) &  0.980 ( 0.057) &  0.990 ( 0.034) &  0.994 ( 0.017) &  0.994 ( 0.011) &  0.996 ( 0.007) \\
	 \hline
	 VH &  0.951 ( 0.094) &  1.004 ( 0.064) &  1.031 ( 0.045) &  1.031 ( 0.023) &  1.001 ( 0.012) &  0.998 ( 0.012) \\
	 \rowcolor{yellow!20}
	    &  1.087 ( 0.082) &  1.004 ( 0.064) &  1.011 ( 0.037) &  1.006 ( 0.018) &  1.001 ( 0.012) &  1.002 ( 0.008) \\
	 \hline
	 VK &  0.965 ( 0.093) &  1.007 ( 0.063) &  1.030 ( 0.045) &  1.030 ( 0.023) &  0.998 ( 0.013) &  1.001 ( 0.008) \\
	 \rowcolor{yellow!20}
	    &  1.080 ( 0.080) &  1.007 ( 0.063) &  1.013 ( 0.037) &  1.008 ( 0.018) &  1.001 ( 0.011) &  1.001 ( 0.008) \\
	 \hline
	 IJ &  0.939 ( 0.087) &  0.969 ( 0.060) &  0.969 ( 0.047) &  0.985 ( 0.023) &  0.948 ( 0.017) &  0.946 ( 0.017) \\
	 \rowcolor{yellow!20}
	    &  0.764 ( 0.079) &  0.814 ( 0.069) &  0.969 ( 0.047) &  0.985 ( 0.023) &  0.992 ( 0.014) &  0.996 ( 0.010) \\
	 \hline
	 IH &  0.956 ( 0.096) &  1.005 ( 0.066) &  1.024 ( 0.049) &  1.036 ( 0.026) &  1.007 ( 0.015) &  1.005 ( 0.014) \\
	 \rowcolor{yellow!20}
	    &  0.856 ( 0.095) &  0.891 ( 0.079) &  1.024 ( 0.049) &  1.013 ( 0.023) &  1.007 ( 0.015) &  1.007 ( 0.010) \\
	 \hline
	 IK &  0.969 ( 0.093) &  1.008 ( 0.064) &  1.025 ( 0.048) &  1.032 ( 0.025) &  1.001 ( 0.015) &  1.005 ( 0.010) \\
	 \rowcolor{yellow!20}
	    &  0.868 ( 0.090) &  0.906 ( 0.076) &  1.025 ( 0.048) &  1.013 ( 0.023) &  1.007 ( 0.014) &  1.005 ( 0.010) \\
	 \hline
	 JH &  0.941 ( 0.104) &  1.008 ( 0.073) &  1.044 ( 0.054) &  1.053 ( 0.034) &  1.032 ( 0.021) &  1.030 ( 0.020) \\
	 \rowcolor{yellow!20}
	    &  0.477 ( 0.133) &  0.654 ( 0.112) &  0.977 ( 0.072) &  1.053 ( 0.034) &  1.032 ( 0.021) &  1.027 ( 0.014) \\
	 \hline
	 JK &  0.969 ( 0.096) &  1.011 ( 0.067) &  1.035 ( 0.049) &  1.038 ( 0.029) &  1.010 ( 0.018) &  1.016 ( 0.012) \\
	 \rowcolor{yellow!20}
	    &  0.678 ( 0.110) &  0.788 ( 0.096) &  1.000 ( 0.060) &  1.038 ( 0.029) &  1.024 ( 0.018) &  1.016 ( 0.012) \\
	 \hline
	 HK &  0.962 ( 0.090) &  0.984 ( 0.063) &  0.998 ( 0.048) &  0.997 ( 0.030) &  0.964 ( 0.024) &  0.976 ( 0.016) \\
	 \rowcolor{yellow!20}
	    &  0.375 ( 0.061) &  0.445 ( 0.064) &  0.703 ( 0.058) &  0.911 ( 0.035) &  0.964 ( 0.024) &  0.976 ( 0.016) \\
	 \hline     \end{tabular}
    }
	\label{tab:symbols}
\end{table}

\begin{align}
\lim_{\lambda \to \infty } \rho^{12}_{\lambda \lambda} = 1
\end{align}

This behavior is particularly evident at longer wavelengths, such as the $K$ band shown in Table~\ref{tab:rho}. The yellow-highlighted cases represent the slopes, $\rho$, obtained using my approach. This is not observed in Madore's approach, as it uses a fixed PW residual for all bands along the $y$-axis, resulting in deviation from slope 1. To illustrate this more clearly, consider the case of an incorrect reddening law, $R_\lambda^{12}$, where $\Delta W_\lambda^{12}$ takes the form:

\begin{align*}
\Delta W^{12}_\lambda = - \delta \mu - \delta R^{12}_\lambda (m_1 - m_2)
\end{align*}
 which affects $\rho$ as follows:

\textbf{Madore (for $(V-I)$):}
\begin{align}
\rho^{VI}_{V \lambda} = \frac{- \delta \mu - \delta A_\lambda}{ - \delta \mu - \delta R^{VI}_V (V - I)}
\end{align}

As the wavelength increases, $\delta A_\lambda$ decreases, while $\delta R_V^{VI}$ remains fixed, causing the resulting ratio to be smaller than 1 even at longer wavelengths.


\textbf{Shubham (for $(V-I)$):}
\begin{align}
\rho^{VI}_{\lambda \lambda} = \frac{- \delta \mu - \delta A_\lambda}{ - \delta \mu - \delta R^{VI}_\lambda (V - I)}
\label{rho_S}
\end{align}

Here, both $\delta A_\lambda$ and $\delta R_\lambda^{VI}$ decrease with increasing wavelength, causing the numerator and denominator to converge, and the slope approaches 1 at longer wavelengths.

Since the deviation between $\rho_{V \lambda}^{VI}$ and $\rho_{\lambda \lambda}^{VI}$ is non-zero, the derived reddening corrections will also differ from those of Madore. The difference between the two reddening errors, prior to distance correction, can be expressed as:

\begin{align*}
    \delta E_{\lambda \lambda}^{VI}(0) -  \delta E_{V \lambda}^{VI}(0) & = \frac{\delta A_{\lambda \lambda}^{VI}(0) - \delta A_{V \lambda}^{VI}(0)}{R^{VI}_\lambda} \\
    & = \frac{\Delta_{\lambda \lambda}^{VI}- \Delta_{V \lambda}^{VI}}{R^{VI}_\lambda} \\
    & = \frac{ \Delta W_\lambda^{VI}  \times \rho_{\lambda \lambda}^{VI} - \Delta W_V^{VI} \times \rho_{V \lambda}^{VI}}{R^{VI}_\lambda} \\
\end{align*}

Determination of distance correction ($\delta \mu^*$) requires scaling factor $(1-\rho^{12}_{\kappa \lambda})$, implies distance correction would also be different. 

\section{Tweaking Distance - Reddening Calibration Method}

Given multiple photometric bands, shorter wavelengths are more sensitive to reddening uncertainties, while distance errors are largely wavelength-independent. To mitigate this, distance and reddening are estimated in two stages. First, the longer-wavelength $IJHK$ bands provide an initial distance - reddening correction via an infrared Wesenheit color, offering robust distance estimates with minimal reddening adjustment. These corrected values are then refined using the shorter-wavelength $BVIJ$ bands and the $(V-I)$ Wesenheit index, which improves reddening accuracy with little effect on distance. The final corrections produce a tightly constrained Galactic Leavitt Law across the $BVIJHK$ bands, with dispersions of $0.087$~mag in $B$ and $0.011$~mag in $K$. The next section details the calibration procedure.

\subsection{Distance correction using IJHK photometry}
To derive the distance correction, I used the longer-wavelength $IJHK$ bands, which are weakly affected by extinction, excluding the more sensitive $B$ and $V$ bands to avoid bias from optical reddening. The correction was estimated using Wesenheit indices based on $(I-J)$, $(I-H)$, $(I-K)$, and $(J-K)$ colors, yielding a robust distance estimate with minimal reddening contamination. 

	\begin{figure}[htp]
		\centering
        \begin{subfigure}[b]{\textwidth}
		\centering
	\includegraphics[width=\linewidth]{./plots/3_deldel/\filename_deldel_S_0IJ_\dis.pdf} 
    \end{subfigure}	        
        \caption{Optical-Infrared Wesenheit based $\Delta M_\lambda$ - $\Delta W^{IJ}_\lambda$ correlation. Notice the relative scatter in between $BV$ and $IJHK$ bands.}
        \label{IJresidue_corr}
	\end{figure}
	\begin{figure}[htp]
		\centering
        \begin{subfigure}[b]{\textwidth}
		\centering
	\includegraphics[width=\linewidth]{./plots/3_deldel/\filename_deldel_S_0JK_\dis.pdf} 
    \end{subfigure}	        
        \caption{Intrared Wesenheit based $\Delta M_\lambda$ - $\Delta W^{JK}_\lambda$ correlation.}
        \label{JKresidue_corr}
	\end{figure}

To illustrate the decoupling of distance and reddening errors, residual-correlation plots for the $(I-J)$ and $(J-K)$ colors are shown. For the colors $(I-J)$, $(I-H)$, $(I-K)$, and $(J-K)$, the Wesenheit magnitudes remain effectively reddening - free even in the presence of intrinsic uncertainties in the extinction law, as described by Equation~\ref{rho_S}. This is supported by the highly consistent values of $\delta \mu$ across all four colors (see Fig.~\ref{dis_correction}).

Among these, $(J-K)$ combines two closely spaced near-infrared bands that are only weakly affected by interstellar extinction. The extinction difference between them is small ($A_J - A_K \approx 0.2$), whereas optical – infrared combinations such as $(I-K)$ or $(I-J)$ are more sensitive to extinction variations. Consequently, $(J-K)$ serves as an almost reddening - independent indicator, enabling a more precise determination of intrinsic luminosity and distance. The $(J-K)$–based distance - calibrated Leavitt Law will subsequently be refined to incorporate the final reddening corrections.

\begin{figure}[htp]
    \centering
    \begin{subfigure}[b]{0.49\textwidth}
        \centering
        \includegraphics[width=\linewidth]{./plots/6_rms/\filename_37_star_SIJ_\dis.pdf}
        \caption{TW CMa}
    \end{subfigure}
    \hfill
    \begin{subfigure}[b]{0.49\textwidth}
        \centering
        \includegraphics[width=\linewidth]{./plots/6_rms/\filename_44_star_SIJ_\dis.pdf}
        \caption{BK Aur}
    \end{subfigure}
        \caption{\small
    $(I-J)$, $(I-H)$, $(I-K)$, and $(J-K)$ based modulus correction. 
    The least dispersion among $I$, $J$, $H$, and $K$ bands gives the correction pair 
    ($\delta \mu$, $\delta E$). 
    The $B$ and $V$ bands are excluded.}
    \label{dis_correction}
\end{figure}
 
	\begin{figure}[h]
		\centering
        \begin{subfigure}[b]{\textwidth}
		\centering
	\includegraphics[width=\linewidth]{./plots//8_result/\filename_PL0_SJK_\dis.pdf} 
    \end{subfigure}	        
        \caption{\small Distance calibration using the $(J - K)$–based Wesenheit index, with the $B$ and $V$ bands excluded, yields minimal reddening effects and high accuracy in distance estimates. The resulting distance correction is then applied uniformly across all bands. | Colored: Original Data | Red : Corrected Data}
        \label{disJK_result}
	\end{figure}
 
\subsection{Reddening correction using BVIJ photometry}

To obtain an accurate reddening correction, the optical and near - infrared bands ($BVIJ$) are used, while the $H$ and $K$ bands are excluded from the analysis. Using the first - order distance –  reddening corrected data, the PL – PLW residual correlation is re - evaluated. Since the remaining distance correction is expected to be minimal, the residual slope in this second - order correction should not approach unity as rapidly as observed in the first-order case.

	\begin{figure}[h]
		\centering
        \begin{subfigure}[b]{\textwidth}
		\centering
	\includegraphics[width=\linewidth]{./plots/3_deldel/\ufilename_deldel_S_0VI_\dis.pdf} 
    \end{subfigure}	        
        \caption{Residual correlation for $V-I$. High variation in slope indicating absence of systematic effect of distance error.}
        \label{VIresidue_corr}
	\end{figure}

\begin{figure}[htp!]
    \centering
    % Row 1
    \begin{subfigure}[b]{0.48\textwidth}
        \centering
        \includegraphics[width=\linewidth]{./plots/6_rms/\ufilename_37_star_SBV_\dis.pdf}
        \caption{TW CMa}
    \end{subfigure}
    \hfill
    \begin{subfigure}[b]{0.48\textwidth}
        \centering
        \includegraphics[width=\linewidth]{./plots/6_rms/\ufilename_44_star_SBV_\dis.pdf}
        \caption{BK Aur}
    \end{subfigure}
     
    % Row 2
    \begin{subfigure}[b]{0.48\textwidth}
        \centering
        \includegraphics[width=\linewidth]{./plots/6_rms/\ufilename_5_star_SBV_\dis.pdf}
        \caption{AD Gem}
    \end{subfigure}
    \hfill
    \begin{subfigure}[b]{0.48\textwidth}
        \centering
        \includegraphics[width=\linewidth]{./plots/6_rms/\ufilename_30_star_SBV_\dis.pdf}
        \caption{RS Cas}
    \end{subfigure}
        \caption{\small    $(B-V)$, $(B-I)$, $(B-J)$, and $(V-I)$ based reddening correction. 
    The $H$ and $K$ bands are excluded. Notice the consistency in reddening and modulus correction accross the colors.}
    \label{rd_correction}
\end{figure}
 
The high dispersion in the residual – slopes for the second-order correction clearly demonstrates the effectiveness of the first-order distance correction, as the systematic trend with a slope of unity is no longer followed with increasing wavelength. It is also noteworthy that the range of PL residuals (y - axis) decreases rapidly across the bands, indicating significant contamination from extinction errors still present. The reddening uncertainty can thus be quantified by identifying the distance solution that minimizes dispersion across the bands.


The robustness of this approach is illustrated for a sample of four stars, as shown in Figure~\ref{rd_correction}. For the given sample of Cepheids, the reddening errors derived from different colors show strong mutual agreement, requiring little to no additional distance correction across all bands. For individual Cepheids, applying these optical-based corrections yields the final calibrated results.

\section{Calibrated Leavitt Law}
Each Cepheid in the dataset is corrected for distance and reddening, yielding calibrated true absolute magnitudes in the $BVIJHK$ bands. This produces a calibrated Leavitt Law, for which the $(J-K)$–based modulus and $(V-I)$–based reddening exhibit the least scatter. 
    \begin{figure}[htp]
		\centering
    \begin{subfigure}[b]{\textwidth}
		\centering
	\includegraphics[width=0.98\linewidth]{./plots/8_result/\ufilename_PL0_SVI_\dis.pdf} 
    \end{subfigure}	
        \caption{\small $BVIJ$ based reddening correction (second - order) estimated with $(V-I)$–based Wesenheit index, applied across all bands to yield the final $BVIJHK$ Leavitt Laws. Background continous line represents original data.}
        \label{SMreddening}
	\end{figure} 

Calibrated Leavitt Law are:
$$B = (-1.844 \pm 0.030) (\log P -1) -3.140 \pm 0.009 $$
$$V = (-2.268 \pm 0.010) (\log P -1) -3.953 \pm 0.003 $$
$$I = (-2.569 \pm 0.015) (\log P -1) -4.738 \pm 0.004 $$
$$J = (-2.786 \pm 0.008) (\log P -1) -5.225 \pm 0.002 $$
$$H = (-2.916 \pm 0.007) (\log P -1) -5.599 \pm 0.002 $$
$$K = (-2.971 \pm 0.004) (\log P -1) -5.660 \pm 0.001 $$

\subsection{Cross Validation with Cruz Cluster Cepheids}

In the dataset, 17 Cepheids are identified as members of clusters with statistical parallaxes measured by \textcite{cruz2023}. A comparison between the calibrated distances of these Cepheids shows a good agreement in reddening value with $\sigma = 0.049$. 

For distance modulus, nine of 17 cluster Cepheid deviated by 0.1 mag only, but five Cepheids (EV Sct, QZ Nor, SV Vul, Y Sct and X Lac) have deviation larger than 0.4 mag.

\begin{figure}[htp]
    \centering
    \begin{subfigure}[b]{0.49\textwidth}
        \centering
        \includegraphics[width=\linewidth]{./plots/9_compare/jesper_cruz_rd.pdf}
        \caption{Reddening correction comparision}
    \end{subfigure}
    \hfill
    \begin{subfigure}[b]{0.49\textwidth}
        \centering
        \includegraphics[width=\linewidth]{./plots/9_compare/jesper_cruz_mu.pdf}
        \caption{Distance modulus comparision}
    \end{subfigure}
        \caption{\small Deviation of distance and reddening with 17 Cluster Cepheids (Cruz, 2023)}
    \label{dis_correction}
\end{figure}
  
\section{Distance to Magellanic Clouds}
 
In the vicinity of the Magellanic Clouds, several Cepheids are located, a subset of which have available VIJK photometry and estimated reddenings. For these stars, I adopt the extinction law from \textcite{wang2023} for both the LMC and SMC, as summarized in Table 1, with $R_V$ values of 3.4 and 2.53, respectively. The sample includes 29 Cepheids in the LMC and 32 in the SMC. Their calibrated Leavitt Law are as follows: Left - LMC and Right - SMC
\[
\small
\begin{minipage}[t]{0.49\textwidth}
\begin{aligned}
V = (-2.707 \pm 0.101) (\log P -1) + 14.296 \pm 0.041 \\
I = (-2.960 \pm 0.076) (\log P -1) + 13.605 \pm 0.030 \\
J = (-3.124 \pm 0.056) (\log P -1) + 13.216 \pm 0.022 \\
K = (-3.226 \pm 0.043) (\log P -1) + 12.783 \pm 0.017 
\end{aligned}
\end{minipage}
\hfill
\begin{minipage}[t]{0.49\textwidth}
\begin{aligned}
V &= (-2.861 \pm 0.139)(\log P - 1) + 15.172 \pm 0.055 \\
I &= (-3.089 \pm 0.102)(\log P - 1) + 14.367 \pm 0.040 \\
J &= (-3.016 \pm 0.088)(\log P - 1) + 13.798 \pm 0.034 \\
K &= (-3.260 \pm 0.071)(\log P - 1) + 13.367 \pm 0.028
\end{aligned}
\end{minipage}
\]

Comparing the intercepts with Galactic Leavitt Law in each band yields distance moduli of $18.365 \pm 0.110$ for the LMC and $19.068 \pm 0.038$ for the SMC when the $VIJK$ bands are analyzed jointly.
 
\begin{figure}[htp]
    \centering
    \includegraphics[width=0.6\linewidth]{./plots/9_compare/LMCSMCmu.pdf}
    \caption{LMC and SMC distance using calibrated VIJK Leavitt Law intercepts}
    \label{fig:wrapped}
\end{figure}

%\bibliographystyle{aasjournal}   % or plainnat, apj, mnras, etc.
%\bibliography{references}        % name of your .bib file (without .bib)

\printbibliography
 
\end{document}
